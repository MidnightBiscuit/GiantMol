
% Default to the notebook output style

    


% Inherit from the specified cell style.




    
\documentclass[11pt]{article}

    
    
    \usepackage[T1]{fontenc}
    % Nicer default font (+ math font) than Computer Modern for most use cases
    \usepackage{mathpazo}

    % Basic figure setup, for now with no caption control since it's done
    % automatically by Pandoc (which extracts ![](path) syntax from Markdown).
    \usepackage{graphicx}
    % We will generate all images so they have a width \maxwidth. This means
    % that they will get their normal width if they fit onto the page, but
    % are scaled down if they would overflow the margins.
    \makeatletter
    \def\maxwidth{\ifdim\Gin@nat@width>\linewidth\linewidth
    \else\Gin@nat@width\fi}
    \makeatother
    \let\Oldincludegraphics\includegraphics
    % Set max figure width to be 80% of text width, for now hardcoded.
    \renewcommand{\includegraphics}[1]{\Oldincludegraphics[width=.8\maxwidth]{#1}}
    % Ensure that by default, figures have no caption (until we provide a
    % proper Figure object with a Caption API and a way to capture that
    % in the conversion process - todo).
    \usepackage{caption}
    \DeclareCaptionLabelFormat{nolabel}{}
    \captionsetup{labelformat=nolabel}

    \usepackage{adjustbox} % Used to constrain images to a maximum size 
    \usepackage{xcolor} % Allow colors to be defined
    \usepackage{enumerate} % Needed for markdown enumerations to work
    \usepackage{geometry} % Used to adjust the document margins
    \usepackage{amsmath} % Equations
    \usepackage{amssymb} % Equations
    \usepackage{textcomp} % defines textquotesingle
    % Hack from http://tex.stackexchange.com/a/47451/13684:
    \AtBeginDocument{%
        \def\PYZsq{\textquotesingle}% Upright quotes in Pygmentized code
    }
    \usepackage{upquote} % Upright quotes for verbatim code
    \usepackage{eurosym} % defines \euro
    \usepackage[mathletters]{ucs} % Extended unicode (utf-8) support
    \usepackage[utf8x]{inputenc} % Allow utf-8 characters in the tex document
    \usepackage{fancyvrb} % verbatim replacement that allows latex
    \usepackage{grffile} % extends the file name processing of package graphics 
                         % to support a larger range 
    % The hyperref package gives us a pdf with properly built
    % internal navigation ('pdf bookmarks' for the table of contents,
    % internal cross-reference links, web links for URLs, etc.)
    \usepackage{hyperref}
    \usepackage{longtable} % longtable support required by pandoc >1.10
    \usepackage{booktabs}  % table support for pandoc > 1.12.2
    \usepackage[inline]{enumitem} % IRkernel/repr support (it uses the enumerate* environment)
    \usepackage[normalem]{ulem} % ulem is needed to support strikethroughs (\sout)
                                % normalem makes italics be italics, not underlines
    

    
    
    % Colors for the hyperref package
    \definecolor{urlcolor}{rgb}{0,.145,.698}
    \definecolor{linkcolor}{rgb}{.71,0.21,0.01}
    \definecolor{citecolor}{rgb}{.12,.54,.11}

    % ANSI colors
    \definecolor{ansi-black}{HTML}{3E424D}
    \definecolor{ansi-black-intense}{HTML}{282C36}
    \definecolor{ansi-red}{HTML}{E75C58}
    \definecolor{ansi-red-intense}{HTML}{B22B31}
    \definecolor{ansi-green}{HTML}{00A250}
    \definecolor{ansi-green-intense}{HTML}{007427}
    \definecolor{ansi-yellow}{HTML}{DDB62B}
    \definecolor{ansi-yellow-intense}{HTML}{B27D12}
    \definecolor{ansi-blue}{HTML}{208FFB}
    \definecolor{ansi-blue-intense}{HTML}{0065CA}
    \definecolor{ansi-magenta}{HTML}{D160C4}
    \definecolor{ansi-magenta-intense}{HTML}{A03196}
    \definecolor{ansi-cyan}{HTML}{60C6C8}
    \definecolor{ansi-cyan-intense}{HTML}{258F8F}
    \definecolor{ansi-white}{HTML}{C5C1B4}
    \definecolor{ansi-white-intense}{HTML}{A1A6B2}

    % commands and environments needed by pandoc snippets
    % extracted from the output of `pandoc -s`
    \providecommand{\tightlist}{%
      \setlength{\itemsep}{0pt}\setlength{\parskip}{0pt}}
    \DefineVerbatimEnvironment{Highlighting}{Verbatim}{commandchars=\\\{\}}
    % Add ',fontsize=\small' for more characters per line
    \newenvironment{Shaded}{}{}
    \newcommand{\KeywordTok}[1]{\textcolor[rgb]{0.00,0.44,0.13}{\textbf{{#1}}}}
    \newcommand{\DataTypeTok}[1]{\textcolor[rgb]{0.56,0.13,0.00}{{#1}}}
    \newcommand{\DecValTok}[1]{\textcolor[rgb]{0.25,0.63,0.44}{{#1}}}
    \newcommand{\BaseNTok}[1]{\textcolor[rgb]{0.25,0.63,0.44}{{#1}}}
    \newcommand{\FloatTok}[1]{\textcolor[rgb]{0.25,0.63,0.44}{{#1}}}
    \newcommand{\CharTok}[1]{\textcolor[rgb]{0.25,0.44,0.63}{{#1}}}
    \newcommand{\StringTok}[1]{\textcolor[rgb]{0.25,0.44,0.63}{{#1}}}
    \newcommand{\CommentTok}[1]{\textcolor[rgb]{0.38,0.63,0.69}{\textit{{#1}}}}
    \newcommand{\OtherTok}[1]{\textcolor[rgb]{0.00,0.44,0.13}{{#1}}}
    \newcommand{\AlertTok}[1]{\textcolor[rgb]{1.00,0.00,0.00}{\textbf{{#1}}}}
    \newcommand{\FunctionTok}[1]{\textcolor[rgb]{0.02,0.16,0.49}{{#1}}}
    \newcommand{\RegionMarkerTok}[1]{{#1}}
    \newcommand{\ErrorTok}[1]{\textcolor[rgb]{1.00,0.00,0.00}{\textbf{{#1}}}}
    \newcommand{\NormalTok}[1]{{#1}}
    
    % Additional commands for more recent versions of Pandoc
    \newcommand{\ConstantTok}[1]{\textcolor[rgb]{0.53,0.00,0.00}{{#1}}}
    \newcommand{\SpecialCharTok}[1]{\textcolor[rgb]{0.25,0.44,0.63}{{#1}}}
    \newcommand{\VerbatimStringTok}[1]{\textcolor[rgb]{0.25,0.44,0.63}{{#1}}}
    \newcommand{\SpecialStringTok}[1]{\textcolor[rgb]{0.73,0.40,0.53}{{#1}}}
    \newcommand{\ImportTok}[1]{{#1}}
    \newcommand{\DocumentationTok}[1]{\textcolor[rgb]{0.73,0.13,0.13}{\textit{{#1}}}}
    \newcommand{\AnnotationTok}[1]{\textcolor[rgb]{0.38,0.63,0.69}{\textbf{\textit{{#1}}}}}
    \newcommand{\CommentVarTok}[1]{\textcolor[rgb]{0.38,0.63,0.69}{\textbf{\textit{{#1}}}}}
    \newcommand{\VariableTok}[1]{\textcolor[rgb]{0.10,0.09,0.49}{{#1}}}
    \newcommand{\ControlFlowTok}[1]{\textcolor[rgb]{0.00,0.44,0.13}{\textbf{{#1}}}}
    \newcommand{\OperatorTok}[1]{\textcolor[rgb]{0.40,0.40,0.40}{{#1}}}
    \newcommand{\BuiltInTok}[1]{{#1}}
    \newcommand{\ExtensionTok}[1]{{#1}}
    \newcommand{\PreprocessorTok}[1]{\textcolor[rgb]{0.74,0.48,0.00}{{#1}}}
    \newcommand{\AttributeTok}[1]{\textcolor[rgb]{0.49,0.56,0.16}{{#1}}}
    \newcommand{\InformationTok}[1]{\textcolor[rgb]{0.38,0.63,0.69}{\textbf{\textit{{#1}}}}}
    \newcommand{\WarningTok}[1]{\textcolor[rgb]{0.38,0.63,0.69}{\textbf{\textit{{#1}}}}}
    
    
    % Define a nice break command that doesn't care if a line doesn't already
    % exist.
    \def\br{\hspace*{\fill} \\* }
    % Math Jax compatability definitions
    \def\gt{>}
    \def\lt{<}
    % Document parameters
    \title{Untitled}
    
    
    

    % Pygments definitions
    
\makeatletter
\def\PY@reset{\let\PY@it=\relax \let\PY@bf=\relax%
    \let\PY@ul=\relax \let\PY@tc=\relax%
    \let\PY@bc=\relax \let\PY@ff=\relax}
\def\PY@tok#1{\csname PY@tok@#1\endcsname}
\def\PY@toks#1+{\ifx\relax#1\empty\else%
    \PY@tok{#1}\expandafter\PY@toks\fi}
\def\PY@do#1{\PY@bc{\PY@tc{\PY@ul{%
    \PY@it{\PY@bf{\PY@ff{#1}}}}}}}
\def\PY#1#2{\PY@reset\PY@toks#1+\relax+\PY@do{#2}}

\expandafter\def\csname PY@tok@w\endcsname{\def\PY@tc##1{\textcolor[rgb]{0.73,0.73,0.73}{##1}}}
\expandafter\def\csname PY@tok@c\endcsname{\let\PY@it=\textit\def\PY@tc##1{\textcolor[rgb]{0.25,0.50,0.50}{##1}}}
\expandafter\def\csname PY@tok@cp\endcsname{\def\PY@tc##1{\textcolor[rgb]{0.74,0.48,0.00}{##1}}}
\expandafter\def\csname PY@tok@k\endcsname{\let\PY@bf=\textbf\def\PY@tc##1{\textcolor[rgb]{0.00,0.50,0.00}{##1}}}
\expandafter\def\csname PY@tok@kp\endcsname{\def\PY@tc##1{\textcolor[rgb]{0.00,0.50,0.00}{##1}}}
\expandafter\def\csname PY@tok@kt\endcsname{\def\PY@tc##1{\textcolor[rgb]{0.69,0.00,0.25}{##1}}}
\expandafter\def\csname PY@tok@o\endcsname{\def\PY@tc##1{\textcolor[rgb]{0.40,0.40,0.40}{##1}}}
\expandafter\def\csname PY@tok@ow\endcsname{\let\PY@bf=\textbf\def\PY@tc##1{\textcolor[rgb]{0.67,0.13,1.00}{##1}}}
\expandafter\def\csname PY@tok@nb\endcsname{\def\PY@tc##1{\textcolor[rgb]{0.00,0.50,0.00}{##1}}}
\expandafter\def\csname PY@tok@nf\endcsname{\def\PY@tc##1{\textcolor[rgb]{0.00,0.00,1.00}{##1}}}
\expandafter\def\csname PY@tok@nc\endcsname{\let\PY@bf=\textbf\def\PY@tc##1{\textcolor[rgb]{0.00,0.00,1.00}{##1}}}
\expandafter\def\csname PY@tok@nn\endcsname{\let\PY@bf=\textbf\def\PY@tc##1{\textcolor[rgb]{0.00,0.00,1.00}{##1}}}
\expandafter\def\csname PY@tok@ne\endcsname{\let\PY@bf=\textbf\def\PY@tc##1{\textcolor[rgb]{0.82,0.25,0.23}{##1}}}
\expandafter\def\csname PY@tok@nv\endcsname{\def\PY@tc##1{\textcolor[rgb]{0.10,0.09,0.49}{##1}}}
\expandafter\def\csname PY@tok@no\endcsname{\def\PY@tc##1{\textcolor[rgb]{0.53,0.00,0.00}{##1}}}
\expandafter\def\csname PY@tok@nl\endcsname{\def\PY@tc##1{\textcolor[rgb]{0.63,0.63,0.00}{##1}}}
\expandafter\def\csname PY@tok@ni\endcsname{\let\PY@bf=\textbf\def\PY@tc##1{\textcolor[rgb]{0.60,0.60,0.60}{##1}}}
\expandafter\def\csname PY@tok@na\endcsname{\def\PY@tc##1{\textcolor[rgb]{0.49,0.56,0.16}{##1}}}
\expandafter\def\csname PY@tok@nt\endcsname{\let\PY@bf=\textbf\def\PY@tc##1{\textcolor[rgb]{0.00,0.50,0.00}{##1}}}
\expandafter\def\csname PY@tok@nd\endcsname{\def\PY@tc##1{\textcolor[rgb]{0.67,0.13,1.00}{##1}}}
\expandafter\def\csname PY@tok@s\endcsname{\def\PY@tc##1{\textcolor[rgb]{0.73,0.13,0.13}{##1}}}
\expandafter\def\csname PY@tok@sd\endcsname{\let\PY@it=\textit\def\PY@tc##1{\textcolor[rgb]{0.73,0.13,0.13}{##1}}}
\expandafter\def\csname PY@tok@si\endcsname{\let\PY@bf=\textbf\def\PY@tc##1{\textcolor[rgb]{0.73,0.40,0.53}{##1}}}
\expandafter\def\csname PY@tok@se\endcsname{\let\PY@bf=\textbf\def\PY@tc##1{\textcolor[rgb]{0.73,0.40,0.13}{##1}}}
\expandafter\def\csname PY@tok@sr\endcsname{\def\PY@tc##1{\textcolor[rgb]{0.73,0.40,0.53}{##1}}}
\expandafter\def\csname PY@tok@ss\endcsname{\def\PY@tc##1{\textcolor[rgb]{0.10,0.09,0.49}{##1}}}
\expandafter\def\csname PY@tok@sx\endcsname{\def\PY@tc##1{\textcolor[rgb]{0.00,0.50,0.00}{##1}}}
\expandafter\def\csname PY@tok@m\endcsname{\def\PY@tc##1{\textcolor[rgb]{0.40,0.40,0.40}{##1}}}
\expandafter\def\csname PY@tok@gh\endcsname{\let\PY@bf=\textbf\def\PY@tc##1{\textcolor[rgb]{0.00,0.00,0.50}{##1}}}
\expandafter\def\csname PY@tok@gu\endcsname{\let\PY@bf=\textbf\def\PY@tc##1{\textcolor[rgb]{0.50,0.00,0.50}{##1}}}
\expandafter\def\csname PY@tok@gd\endcsname{\def\PY@tc##1{\textcolor[rgb]{0.63,0.00,0.00}{##1}}}
\expandafter\def\csname PY@tok@gi\endcsname{\def\PY@tc##1{\textcolor[rgb]{0.00,0.63,0.00}{##1}}}
\expandafter\def\csname PY@tok@gr\endcsname{\def\PY@tc##1{\textcolor[rgb]{1.00,0.00,0.00}{##1}}}
\expandafter\def\csname PY@tok@ge\endcsname{\let\PY@it=\textit}
\expandafter\def\csname PY@tok@gs\endcsname{\let\PY@bf=\textbf}
\expandafter\def\csname PY@tok@gp\endcsname{\let\PY@bf=\textbf\def\PY@tc##1{\textcolor[rgb]{0.00,0.00,0.50}{##1}}}
\expandafter\def\csname PY@tok@go\endcsname{\def\PY@tc##1{\textcolor[rgb]{0.53,0.53,0.53}{##1}}}
\expandafter\def\csname PY@tok@gt\endcsname{\def\PY@tc##1{\textcolor[rgb]{0.00,0.27,0.87}{##1}}}
\expandafter\def\csname PY@tok@err\endcsname{\def\PY@bc##1{\setlength{\fboxsep}{0pt}\fcolorbox[rgb]{1.00,0.00,0.00}{1,1,1}{\strut ##1}}}
\expandafter\def\csname PY@tok@kc\endcsname{\let\PY@bf=\textbf\def\PY@tc##1{\textcolor[rgb]{0.00,0.50,0.00}{##1}}}
\expandafter\def\csname PY@tok@kd\endcsname{\let\PY@bf=\textbf\def\PY@tc##1{\textcolor[rgb]{0.00,0.50,0.00}{##1}}}
\expandafter\def\csname PY@tok@kn\endcsname{\let\PY@bf=\textbf\def\PY@tc##1{\textcolor[rgb]{0.00,0.50,0.00}{##1}}}
\expandafter\def\csname PY@tok@kr\endcsname{\let\PY@bf=\textbf\def\PY@tc##1{\textcolor[rgb]{0.00,0.50,0.00}{##1}}}
\expandafter\def\csname PY@tok@bp\endcsname{\def\PY@tc##1{\textcolor[rgb]{0.00,0.50,0.00}{##1}}}
\expandafter\def\csname PY@tok@fm\endcsname{\def\PY@tc##1{\textcolor[rgb]{0.00,0.00,1.00}{##1}}}
\expandafter\def\csname PY@tok@vc\endcsname{\def\PY@tc##1{\textcolor[rgb]{0.10,0.09,0.49}{##1}}}
\expandafter\def\csname PY@tok@vg\endcsname{\def\PY@tc##1{\textcolor[rgb]{0.10,0.09,0.49}{##1}}}
\expandafter\def\csname PY@tok@vi\endcsname{\def\PY@tc##1{\textcolor[rgb]{0.10,0.09,0.49}{##1}}}
\expandafter\def\csname PY@tok@vm\endcsname{\def\PY@tc##1{\textcolor[rgb]{0.10,0.09,0.49}{##1}}}
\expandafter\def\csname PY@tok@sa\endcsname{\def\PY@tc##1{\textcolor[rgb]{0.73,0.13,0.13}{##1}}}
\expandafter\def\csname PY@tok@sb\endcsname{\def\PY@tc##1{\textcolor[rgb]{0.73,0.13,0.13}{##1}}}
\expandafter\def\csname PY@tok@sc\endcsname{\def\PY@tc##1{\textcolor[rgb]{0.73,0.13,0.13}{##1}}}
\expandafter\def\csname PY@tok@dl\endcsname{\def\PY@tc##1{\textcolor[rgb]{0.73,0.13,0.13}{##1}}}
\expandafter\def\csname PY@tok@s2\endcsname{\def\PY@tc##1{\textcolor[rgb]{0.73,0.13,0.13}{##1}}}
\expandafter\def\csname PY@tok@sh\endcsname{\def\PY@tc##1{\textcolor[rgb]{0.73,0.13,0.13}{##1}}}
\expandafter\def\csname PY@tok@s1\endcsname{\def\PY@tc##1{\textcolor[rgb]{0.73,0.13,0.13}{##1}}}
\expandafter\def\csname PY@tok@mb\endcsname{\def\PY@tc##1{\textcolor[rgb]{0.40,0.40,0.40}{##1}}}
\expandafter\def\csname PY@tok@mf\endcsname{\def\PY@tc##1{\textcolor[rgb]{0.40,0.40,0.40}{##1}}}
\expandafter\def\csname PY@tok@mh\endcsname{\def\PY@tc##1{\textcolor[rgb]{0.40,0.40,0.40}{##1}}}
\expandafter\def\csname PY@tok@mi\endcsname{\def\PY@tc##1{\textcolor[rgb]{0.40,0.40,0.40}{##1}}}
\expandafter\def\csname PY@tok@il\endcsname{\def\PY@tc##1{\textcolor[rgb]{0.40,0.40,0.40}{##1}}}
\expandafter\def\csname PY@tok@mo\endcsname{\def\PY@tc##1{\textcolor[rgb]{0.40,0.40,0.40}{##1}}}
\expandafter\def\csname PY@tok@ch\endcsname{\let\PY@it=\textit\def\PY@tc##1{\textcolor[rgb]{0.25,0.50,0.50}{##1}}}
\expandafter\def\csname PY@tok@cm\endcsname{\let\PY@it=\textit\def\PY@tc##1{\textcolor[rgb]{0.25,0.50,0.50}{##1}}}
\expandafter\def\csname PY@tok@cpf\endcsname{\let\PY@it=\textit\def\PY@tc##1{\textcolor[rgb]{0.25,0.50,0.50}{##1}}}
\expandafter\def\csname PY@tok@c1\endcsname{\let\PY@it=\textit\def\PY@tc##1{\textcolor[rgb]{0.25,0.50,0.50}{##1}}}
\expandafter\def\csname PY@tok@cs\endcsname{\let\PY@it=\textit\def\PY@tc##1{\textcolor[rgb]{0.25,0.50,0.50}{##1}}}

\def\PYZbs{\char`\\}
\def\PYZus{\char`\_}
\def\PYZob{\char`\{}
\def\PYZcb{\char`\}}
\def\PYZca{\char`\^}
\def\PYZam{\char`\&}
\def\PYZlt{\char`\<}
\def\PYZgt{\char`\>}
\def\PYZsh{\char`\#}
\def\PYZpc{\char`\%}
\def\PYZdl{\char`\$}
\def\PYZhy{\char`\-}
\def\PYZsq{\char`\'}
\def\PYZdq{\char`\"}
\def\PYZti{\char`\~}
% for compatibility with earlier versions
\def\PYZat{@}
\def\PYZlb{[}
\def\PYZrb{]}
\makeatother


    % Exact colors from NB
    \definecolor{incolor}{rgb}{0.0, 0.0, 0.5}
    \definecolor{outcolor}{rgb}{0.545, 0.0, 0.0}



    
    % Prevent overflowing lines due to hard-to-break entities
    \sloppy 
    % Setup hyperref package
    \hypersetup{
      breaklinks=true,  % so long urls are correctly broken across lines
      colorlinks=true,
      urlcolor=urlcolor,
      linkcolor=linkcolor,
      citecolor=citecolor,
      }
    % Slightly bigger margins than the latex defaults
    
    \geometry{verbose,tmargin=1in,bmargin=1in,lmargin=1in,rmargin=1in}
    
    

    \begin{document}
    
    
    \maketitle
    
    

    
    \hypertarget{v---calcul-mathuxe9matique}{%
\section{V - Calcul mathématique}\label{v---calcul-mathuxe9matique}}

    \begin{Verbatim}[commandchars=\\\{\}]
{\color{incolor}In [{\color{incolor}58}]:} \PY{c+c1}{\PYZsh{} V/1}
         
         \PY{c+c1}{\PYZsh{} Tout d\PYZsq{}abord il faut saisir une date}
         \PY{c+c1}{\PYZsh{} En séparant jour mois et année dans}
         \PY{c+c1}{\PYZsh{} Trois variables distinctes.}
         \PY{c+c1}{\PYZsh{} Ensuite on détermine si l\PYZsq{}année est bissextile.}
         \PY{c+c1}{\PYZsh{} Puis on détermine la longueur du mois.}
         \PY{c+c1}{\PYZsh{} Puis selon le jouron calcule la date du lendemain.}
         
         \PY{c+c1}{\PYZsh{} saisir uniquement des chiffres}
         \PY{n+nb}{print}\PY{p}{(}\PY{l+s+s1}{\PYZsq{}}\PY{l+s+s1}{saisir le jour}\PY{l+s+s1}{\PYZsq{}}\PY{p}{)}
         \PY{n}{jour} \PY{o}{=} \PY{n+nb}{int}\PY{p}{(}\PY{n+nb}{input}\PY{p}{(}\PY{p}{)}\PY{p}{)}
         \PY{n+nb}{print}\PY{p}{(}\PY{l+s+s1}{\PYZsq{}}\PY{l+s+s1}{saisir le mois}\PY{l+s+s1}{\PYZsq{}}\PY{p}{)}
         \PY{n}{mois} \PY{o}{=} \PY{n+nb}{int}\PY{p}{(}\PY{n+nb}{input}\PY{p}{(}\PY{p}{)}\PY{p}{)}
         \PY{n+nb}{print}\PY{p}{(}\PY{l+s+s1}{\PYZsq{}}\PY{l+s+s1}{saisir l}\PY{l+s+s1}{\PYZsq{}}\PY{l+s+s1}{\PYZsq{}}\PY{l+s+s1}{année}\PY{l+s+s1}{\PYZsq{}}\PY{p}{)}
         \PY{n}{annee} \PY{o}{=} \PY{n+nb}{int}\PY{p}{(}\PY{n+nb}{input}\PY{p}{(}\PY{p}{)}\PY{p}{)}
         
         \PY{n+nb}{print}\PY{p}{(}\PY{l+s+s1}{\PYZsq{}}\PY{l+s+s1}{Date saisie :}\PY{l+s+s1}{\PYZsq{}}\PY{p}{,}\PY{n}{jour}\PY{p}{,}\PY{l+s+s1}{\PYZsq{}}\PY{l+s+s1}{/}\PY{l+s+s1}{\PYZsq{}}\PY{p}{,}\PY{n}{mois}\PY{p}{,}\PY{l+s+s1}{\PYZsq{}}\PY{l+s+s1}{/}\PY{l+s+s1}{\PYZsq{}}\PY{p}{,}\PY{n}{annee}\PY{p}{)}
         
         \PY{c+c1}{\PYZsh{} détermination année bissextile ou non}
         \PY{k}{if} \PY{n}{annee}\PY{o}{\PYZpc{}}\PY{k}{4} == 0 and annee\PYZpc{}100!=0: \PYZsh{} si annee multiple de 4 mais pas de 100
             \PY{n}{bissextile} \PY{o}{=} \PY{l+m+mi}{1} \PY{c+c1}{\PYZsh{} annee bissextile}
         \PY{k}{elif} \PY{n}{annee}\PY{o}{\PYZpc{}}\PY{k}{400} == 0: \PYZsh{} ou si annee multiple de 400
             \PY{n}{bissextile} \PY{o}{=} \PY{l+m+mi}{1} \PY{c+c1}{\PYZsh{} annee bissextile}
         \PY{k}{else}\PY{p}{:}
             \PY{n}{bissextile} \PY{o}{=} \PY{l+m+mi}{0} \PY{c+c1}{\PYZsh{} sinon annee non bissextile}
             
         \PY{n+nb}{print}\PY{p}{(}\PY{l+s+s1}{\PYZsq{}}\PY{l+s+s1}{Annee bissextile ? }\PY{l+s+s1}{\PYZsq{}}\PY{p}{,}\PY{n}{bissextile}\PY{p}{)}
         
         \PY{c+c1}{\PYZsh{} determination longueur du mois}
         \PY{k}{if} \PY{n}{mois} \PY{o}{==} \PY{l+m+mi}{2}\PY{p}{:}
             \PY{n}{longueur} \PY{o}{=} \PY{l+m+mi}{28} \PY{o}{+} \PY{n}{bissextile}
         \PY{k}{elif} \PY{n}{mois}\PY{o}{==}\PY{l+m+mi}{1} \PY{o+ow}{or} \PY{n}{mois}\PY{o}{==}\PY{l+m+mi}{3} \PY{o+ow}{or} \PY{n}{mois}\PY{o}{==}\PY{l+m+mi}{5} \PY{o+ow}{or} \PY{n}{mois}\PY{o}{==}\PY{l+m+mi}{7} \PY{o+ow}{or} \PY{n}{mois}\PY{o}{==}\PY{l+m+mi}{8} \PY{o+ow}{or} \PY{n}{mois}\PY{o}{==}\PY{l+m+mi}{10} \PY{o+ow}{or} \PY{n}{mois}\PY{o}{==}\PY{l+m+mi}{12}\PY{p}{:}
             \PY{n}{longueur} \PY{o}{=} \PY{l+m+mi}{31}
         \PY{k}{else}\PY{p}{:}
             \PY{n}{longueur} \PY{o}{=} \PY{l+m+mi}{30}
         
         \PY{c+c1}{\PYZsh{} détermination du jour du lendemain}
         \PY{n}{annee\PYZus{}demain} \PY{o}{=} \PY{n}{annee} \PY{c+c1}{\PYZsh{} on part de l\PYZsq{}année en cours}
         
         \PY{k}{if} \PY{n}{jour} \PY{o}{\PYZlt{}} \PY{n}{longueur}\PY{p}{:} \PY{c+c1}{\PYZsh{} si le jour actuel est inférieur à la longueur du mois}
             \PY{c+c1}{\PYZsh{} c\PYZsq{}est à dire si on n\PYZsq{}est pas le dernier jour du mois}
             \PY{n}{mois\PYZus{}demain} \PY{o}{=} \PY{n}{mois}
             \PY{n}{jour\PYZus{}demain} \PY{o}{=} \PY{n}{jour} \PY{o}{+} \PY{l+m+mi}{1} \PY{c+c1}{\PYZsh{} le jour d\PYZsq{}après}
         \PY{k}{else}\PY{p}{:} \PY{c+c1}{\PYZsh{} sinon c\PYZsq{}est qu\PYZsq{}on est le dernier jour du mois}
             \PY{n}{mois\PYZus{}demain} \PY{o}{=} \PY{n}{mois} \PY{o}{+} \PY{l+m+mi}{1} \PY{c+c1}{\PYZsh{} on commence un nouveau mois}
             \PY{n}{jour\PYZus{}demain} \PY{o}{=} \PY{l+m+mi}{1}
             \PY{k}{if} \PY{n}{mois\PYZus{}demain} \PY{o}{\PYZgt{}} \PY{l+m+mi}{12}\PY{p}{:} \PY{c+c1}{\PYZsh{} mais si on est au dernier mois}
                                  \PY{c+c1}{\PYZsh{} en changeant de mois on change d\PYZsq{}année}
                 \PY{n}{annee\PYZus{}demain} \PY{o}{=} \PY{n}{annee} \PY{o}{+} \PY{l+m+mi}{1} \PY{c+c1}{\PYZsh{} la nouvelle année}
                 \PY{n}{mois\PYZus{}demain} \PY{o}{=} \PY{l+m+mi}{1} \PY{c+c1}{\PYZsh{} janvier}
                 \PY{n+nb}{print}\PY{p}{(}\PY{l+s+s1}{\PYZsq{}}\PY{l+s+s1}{BONNE ANNEE !}\PY{l+s+s1}{\PYZsq{}}\PY{p}{)}
         
         \PY{n+nb}{print}\PY{p}{(}\PY{l+s+s1}{\PYZsq{}}\PY{l+s+s1}{Date demain :}\PY{l+s+s1}{\PYZsq{}}\PY{p}{,}\PY{n}{jour\PYZus{}demain}\PY{p}{,}\PY{l+s+s1}{\PYZsq{}}\PY{l+s+s1}{/}\PY{l+s+s1}{\PYZsq{}}\PY{p}{,}\PY{n}{mois\PYZus{}demain}\PY{p}{,}\PY{l+s+s1}{\PYZsq{}}\PY{l+s+s1}{/}\PY{l+s+s1}{\PYZsq{}}\PY{p}{,}\PY{n}{annee\PYZus{}demain}\PY{p}{)}
\end{Verbatim}


\begin{Verbatim}[commandchars=\\\{\}]
{\color{outcolor}Out[{\color{outcolor}58}]:} [<matplotlib.lines.Line2D at 0x7f75a51af198>]
\end{Verbatim}
            
    \begin{Verbatim}[commandchars=\\\{\}]
{\color{incolor}In [{\color{incolor}1}]:} \PY{c+c1}{\PYZsh{} V/2}
        
        \PY{n}{entier1} \PY{o}{=} \PY{n+nb}{input}\PY{p}{(}\PY{p}{)}
        \PY{n}{entier2} \PY{o}{=} \PY{n+nb}{input}\PY{p}{(}\PY{p}{)}
        \PY{n}{identique} \PY{o}{=} \PY{l+m+mi}{0}
        \PY{k}{for} \PY{n}{k} \PY{o+ow}{in} \PY{n+nb}{range}\PY{p}{(}\PY{l+m+mi}{8}\PY{p}{)}\PY{p}{:}
            \PY{k}{if} \PY{n}{entier1}\PY{p}{[}\PY{n}{k}\PY{p}{]}\PY{o}{==}\PY{n}{entier2}\PY{p}{[}\PY{n}{k}\PY{p}{]}\PY{p}{:}
                \PY{n}{identique} \PY{o}{+}\PY{o}{=}\PY{l+m+mi}{1}
        \PY{n+nb}{print}\PY{p}{(}\PY{n}{identique}\PY{p}{)}
\end{Verbatim}


    \begin{Verbatim}[commandchars=\\\{\}]
12345678
08455487
1

    \end{Verbatim}

    \begin{Verbatim}[commandchars=\\\{\}]
{\color{incolor}In [{\color{incolor}7}]:} \PY{c+c1}{\PYZsh{} V/3}
        
        \PY{k+kn}{from} \PY{n+nn}{math} \PY{k}{import} \PY{n}{sqrt}
        \PY{k+kn}{from} \PY{n+nn}{math} \PY{k}{import} \PY{n}{pi}
        
        \PY{c+c1}{\PYZsh{} La somme sera contenue dans la variable appelée S\PYZus{}new}
        \PY{c+c1}{\PYZsh{} Il va falloir calculer la valeur de S en additionnant}
        \PY{c+c1}{\PYZsh{} les 1/n². Il faut donc calculer 1/n² puis l\PYZsq{}ajouter à}
        \PY{c+c1}{\PYZsh{} la variable S\PYZus{}new, dans une boucle while pour répéter}
        \PY{c+c1}{\PYZsh{} l\PYZsq{}opération autant que voulu.}
        \PY{c+c1}{\PYZsh{} Le calcul doit s\PYZsq{}arrêter lorsque la différence entre}
        \PY{c+c1}{\PYZsh{} la nouvelle valeur de S\PYZus{}new et l\PYZsq{}ancienne   (appelée }
        \PY{c+c1}{\PYZsh{} S\PYZus{}old) est inférieure à une certaine valeur (epsilon).}
        \PY{c+c1}{\PYZsh{} Pour cela nous utilisons une boucle while et calculons}
        \PY{c+c1}{\PYZsh{} A chaque fois cette différence.}
        
        
        \PY{n}{S\PYZus{}old} \PY{o}{=} \PY{l+m+mi}{0}      \PY{c+c1}{\PYZsh{} Valeur de la somme à l\PYZsq{}étape n}
        \PY{n}{S\PYZus{}new} \PY{o}{=} \PY{l+m+mi}{0}      \PY{c+c1}{\PYZsh{} Valeur de la somme à l\PYZsq{}étape n\PYZhy{}1}
        \PY{n}{n} \PY{o}{=} \PY{l+m+mi}{0}          \PY{c+c1}{\PYZsh{} n le nombre de passage dans la boucle}
                       \PY{c+c1}{\PYZsh{} Valeur va évoluer à chaque passage dans}
                       \PY{c+c1}{\PYZsh{} la boucle while}
        \PY{n}{epsilon} \PY{o}{=} \PY{l+m+mf}{1e\PYZhy{}7} \PY{c+c1}{\PYZsh{} valeur epsilon choisie}
                       \PY{c+c1}{\PYZsh{} changer pour modifier la précision}
        \PY{n}{difference} \PY{o}{=} \PY{l+m+mi}{1}
        
        \PY{k}{while} \PY{n}{difference} \PY{o}{\PYZgt{}} \PY{n}{epsilon}\PY{p}{:}
            \PY{n}{n} \PY{o}{+}\PY{o}{=} \PY{l+m+mi}{1}            \PY{c+c1}{\PYZsh{} on incrémente de 1 la valeur}
            \PY{n}{S\PYZus{}old} \PY{o}{=} \PY{n}{S\PYZus{}new}     \PY{c+c1}{\PYZsh{} on défini comme ancienne valeur}
                              \PY{c+c1}{\PYZsh{} ce qui était précédement la nouvelle}
            \PY{n}{S\PYZus{}new} \PY{o}{=} \PY{n}{S\PYZus{}new} \PY{o}{+} \PY{l+m+mi}{1}\PY{o}{/}\PY{p}{(}\PY{n}{n}\PY{o}{*}\PY{o}{*}\PY{l+m+mi}{2}\PY{p}{)} \PY{c+c1}{\PYZsh{} on calcule la nouvelle valeur de la somme}
                                     \PY{c+c1}{\PYZsh{} en additionnant 1/n**2 à S\PYZus{}new}
            \PY{n}{difference} \PY{o}{=} \PY{n}{S\PYZus{}new} \PY{o}{\PYZhy{}} \PY{n}{S\PYZus{}old}
        
        \PY{n+nb}{print}\PY{p}{(}\PY{n}{n}\PY{p}{,}\PY{l+s+s1}{\PYZsq{}}\PY{l+s+s1}{itérations dans la boule while}\PY{l+s+s1}{\PYZsq{}}\PY{p}{)}     \PY{c+c1}{\PYZsh{} combien d\PYZsq{}itérations effectuées}
        \PY{n+nb}{print}\PY{p}{(}\PY{l+s+s1}{\PYZsq{}}\PY{l+s+s1}{S\PYZus{}new =}\PY{l+s+s1}{\PYZsq{}}\PY{p}{,}\PY{n}{S\PYZus{}new}\PY{p}{)}
        \PY{n}{mon\PYZus{}pi} \PY{o}{=} \PY{n}{sqrt}\PY{p}{(}\PY{l+m+mi}{6}\PY{o}{*}\PY{n}{S\PYZus{}new}\PY{p}{)}
        \PY{n+nb}{print}\PY{p}{(}\PY{l+s+s1}{\PYZsq{}}\PY{l+s+s1}{mon pi calculé vaut}\PY{l+s+s1}{\PYZsq{}}\PY{p}{,}\PY{n}{mon\PYZus{}pi}\PY{p}{)}
        \PY{n+nb}{print}\PY{p}{(}\PY{l+s+s1}{\PYZsq{}}\PY{l+s+s1}{ pi vaut en réalité}\PY{l+s+s1}{\PYZsq{}}\PY{p}{,}\PY{n}{pi}\PY{p}{,}\PY{l+s+s1}{\PYZsq{}}\PY{l+s+s1}{...}\PY{l+s+s1}{\PYZsq{}}\PY{p}{)}
\end{Verbatim}


    \begin{Verbatim}[commandchars=\\\{\}]
3163 itérations dans la boule while
S\_new = 1.644617961271602
mon pi calculé vaut 3.1412907804960706
 pi vaut en réalité 3.141592653589793 {\ldots}

    \end{Verbatim}


    % Add a bibliography block to the postdoc
    
    
    
    \end{document}
